This chapter covers the motivation for this thesis, along with research goals, research
questions, and a list of contributions.

\section{Motivation}
%todo Brief overview of APR
Modern software systems continuously evolve with inevitable bugs due to feature deprecation, new features added, and refactoring. These bugs are widely known as a destructive problem, generating costs up to trillions of dollars every year.%todo: \cite{todo}.

Bugs and errors in code are a recurring problem in the software industry, leading to collateral effects of multiple sizes. %\cite{todo} 
As a consequence, there is a substantial amount of time spent only to solve these problems in an efficient way with litle or no human intervention.

In APR research of the last decade, most of traditional methods include search-based, constraint-based, and template-based approaches.%todo cite
These studies led to the development of methods such of SimFix, VarFix, SearchRepair, and Tfix.
More recent studies use machine learning and deep learning techniques were employed to improve program repair tasks.
% cite
With the growth of Large Language Models (LLMs), the Software Engineering area was significantly transformed, specially due to Code LLMs. These models are either pretrained or finetuned on programming languages, directly affecting areas such as code summarization, code generation, bug detection and also APR.


%todo Challenges in traditional methods

%todo Introduction to LLMs as potential solution



\section{Problem Statement}
%todo Description of the problem, and limitations of traditional methods

%todo Description of the problem, and limitations of LLMs. Gap between sytactical and semantic correctness.

\section{Objective and Research Questions} 
 %todo Change to Objective and Research Questions
%todo Objective of the thesis
Esse projeto busca estudar diferentes técnicas de prompt engineering, em realizar tarefas de Patch Correction dentro do campo de Automated Program Repair (APR).


\textbf{RQ1: Comparison between specialized code models vs general-purpose models?}

\textbf{RQ2: Does style based prompt improve the overall effectiveness of responses?}

\textbf{RQ3: Does system prompt improve performances of APR tasks?}


\section{Contribution}