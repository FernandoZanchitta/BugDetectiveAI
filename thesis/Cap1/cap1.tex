\section{Introduction}
%todo Brief overview of APR

Sistemas modernos de software continuamente evoluem com bugs inevitáveis devido a deprecação de features, adição de novas funcionalidades, e refatoração. Esses bugs são amplamente reconhecidos como um problema destrutivo, gerando custos na casa de Trilhões de dólares anualmente \ref(todo).

Erros e bugs em códigos são um problema recorrente na indústria de software, e podem trazer efeitos colaterais das mais diversas ordens de grandeza \ref(todo). Consequentemente, existe um esforço oneroso para que engenheiros de software tratem esses problemas de forma eficiente e adequada.

Automatic Program Repair (APR) é um subcampo da programação voltado para reparação de erros e vulnerabilidades de códigos, programas ou sistemas de forma estruturada e automatizada. Essa área de estudo foca no conjunto de técnicas que possibilitam a correção desses problemas de forma eficiente e adequada. 

Não importa qual método de identificação e correção de bugs é utilizado, o objetivo é encontrar um patch que mude o programa apropriadamente de forma a corrigir o problema.

Na última década métodos de APR tem sido extensivamente pesquisados, e podem ser categorizados em grupos, incluindo:
\begin{itemize}
    \item Métodos heuristics.
    \item Métodos Baseados em constraints.
    \item Métodos Baseados em patterns.
\end{itemize}


%todo Challenges in traditional methods

%todo Introduction to LLMs as potential solution



\section{Problem Statement}
%todo Description of the problem, and limitations of traditional methods

%todo Description of the problem, and limitations of LLMs. Gap between sytactical and semantic correctness.

\section{Objective and Research Questions} 
 %todo Change to Objective and Research Questions
%todo Objective of the thesis
